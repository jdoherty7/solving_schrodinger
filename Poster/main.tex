%%%%%%%%%%%%%%%%%%%%%%%%%%%%%%%%%%%%%%%%%
% Jacobs Landscape Poster
% LaTeX Template
% Version 1.0 (29/03/13)
%
% Created by:
% Computational Physics and Biophysics Group, Jacobs University
% https://teamwork.jacobs-university.de:8443/confluence/display/CoPandBiG/LaTeX+Poster
% 
% Further modified by:
% Nathaniel Johnston (nathaniel@njohnston.ca)
%
% This template has been downloaded from:
% http://www.LaTeXTemplates.com
%
% License:
% CC BY-NC-SA 3.0 (http://creativecommons.org/licenses/by-nc-sa/3.0/)
%
%%%%%%%%%%%%%%%%%%%%%%%%%%%%%%%%%%%%%%%%%

%----------------------------------------------------------------------------------------
%	PACKAGES AND OTHER DOCUMENT CONFIGURATIONS
%----------------------------------------------------------------------------------------

\documentclass[final]{beamer}
\usepackage[scale=0.78,size=a1]{beamerposter} % Use the beamerposter package for laying out the poster
% \usepackage[size=ansiD]{beamerposter} % Use the beamerposter package for laying out the poster

\usetheme{confposter} % Use the confposter theme supplied with this template

\setbeamercolor{block title}{fg=ngreen,bg=white} % Colors of the block titles
\setbeamercolor{block body}{fg=black,bg=white} % Colors of the body of blocks
\setbeamercolor{block alerted title}{fg=white,bg=dblue!70} % Colors of the highlighted block titles
\setbeamercolor{block alerted body}{fg=black,bg=dblue!10} % Colors of the body of highlighted blocks
% Many more colors are available for use in beamerthemeconfposter.sty

%-----------------------------------------------------------
% Define the column widths and overall poster size
% To set effective sepwid, onecolwid and twocolwid values, first choose how many columns you want and how much separation you want between columns
% In this template, the separation width chosen is 0.024 of the paper width and a 4-column layout
% onecolwid should therefore be (1-(# of columns+1)*sepwid)/# of columns e.g. (1-(4+1)*0.024)/4 = 0.22
% Set twocolwid to be (2*onecolwid)+sepwid = 0.464
% Set threecolwid to be (3*onecolwid)+2*sepwid = 0.708

\newlength{\sepwid}
\newlength{\onecolwid}
\newlength{\twocolwid}
\newlength{\threecolwid}
% \setlength{\paperwidth}{33.1in} % A0 width: 46.8in
% \setlength{\paperheight}{23.4in} % A0 height: 33.1in
\setlength{\paperwidth}{34in} % A0 width: 46.8in
\setlength{\paperheight}{22in} % A0 height: 33.1in
\setlength{\sepwid}{0.024\paperwidth} % Separation width (white space) between columns
\setlength{\onecolwid}{0.22\paperwidth} % Width of one column
\setlength{\twocolwid}{0.464\paperwidth} % Width of two columns
\setlength{\threecolwid}{0.708\paperwidth} % Width of three columns
\setlength{\topmargin}{-0.5in} % Reduce the top margin size
%-----------------------------------------------------------

\usepackage{graphicx}  % Required for including images

\usepackage{booktabs} % Top and bottom rules for tables

\usepackage{braket}

%----------------------------------------------------------------------------------------
%	TITLE SECTION 
%----------------------------------------------------------------------------------------

\title{Numerical Time-Dependent Schr\"{o}dinger Equation} % Poster title

\author{John Doherty and Ricardo Castro Yarza} % Author(s)

\institute{University of Illinois at Urbana-Champaign\\Deptartment of Computer Science and Department of Physics} % Institution(s)

%----------------------------------------------------------------------------------------

\begin{document}

\addtobeamertemplate{block end}{}{\vspace*{2ex}} % White space under blocks
\addtobeamertemplate{block alerted end}{}{\vspace*{2ex}} % White space under highlighted (alert) blocks

\setlength{\belowcaptionskip}{2ex} % White space under figures
\setlength\belowdisplayshortskip{2ex} % White space under equations

\begin{frame}[t] % The whole poster is enclosed in one beamer frame

\begin{columns}[t] % The whole poster consists of three major columns, the second of which is split into two columns twice - the [t] option aligns each column's content to the top

\begin{column}{\sepwid}\end{column} % Empty spacer column

\begin{column}{\onecolwid} % The first column

%----------------------------------------------------------------------------------------
%	OBJECTIVES
%----------------------------------------------------------------------------------------

\begin{alertblock}{Abstract}
The Time-Dependent Schr\"{o}dinger Equation (TDSE) describes the quantum state of a particle in the non relativistic regime. The lack of analytical solutions for most cases of interest motivates its study with numerical methods. We present a numerical scheme for the TDSE using Finite Element Method and Crank-Nicolson.
\end{alertblock}

%----------------------------------------------------------------------------------------
%	INTRODUCTION
%----------------------------------------------------------------------------------------

\begin{block}{Introduction}
The TDSE is, in natural units ($\hbar=1$):
\begin{equation}
i\partial_{t}\ket{\psi}=\hat{H}\ket{\psi}
\end{equation}
and takes the following form in position basis:
\begin{equation}
i\partial_{t}\psi=\left[-\frac{1}{2m}\nabla^{2}+V\left(\mathbf{r}\right)\right]\psi
\label{SchrodPosBasis}
\end{equation}
The probability density for the particle's position is given by:
\begin{equation}
\left|\psi\left(\mathbf{r},t\right)\right|^{2}
\end{equation}
\end{block}
\begin{block}{Numerical method}
Due to its geometric flexibility, finite element method with quadratic elements was preferred. The chosen timestepping method is Crank-Nicolson because it is unitary, meaning the norm of the solution is preserved and thus is probability. Applying these two methods requires computing the potential-weighted mass matrix, $C$, given by:
\begin{equation}
C_{ij} = \int\phi_{i}\phi_{j}V\left(\mathbf{r}\right)\,\mathrm{d}\mathbf{x}
\end{equation}
In terms of the global finite element matrices $A$ (stiffness), $B$ (mass), potential-weighted mass $C$, the timestepping operator (equivalent to the quantum mechanical propagator in the position basis):
	%\begin{equation}
	%iv_{i}\left(\int\phi_{i}\phi_{j}\,\mathrm{d}\mathbf{x}\right)\dot{\psi}_{j}=\frac{1}{2m}v_{i}\left(\int\nabla\phi_{i}\nabla\phi_{j}\,\mathrm{d}\mathbf{x}\right)\dot{\psi}_{j}
	%\end{equation}
\begin{equation}
U=\left[B+\frac{i\Delta t}{2}\left(\frac{A}{2m}+C\right)\right]^{-1}\left[B+\frac{i\Delta t}{2}\left(\frac{A}{2m}-C\right)\right]
\end{equation}
which satisfies:
\begin{equation}
\psi^{n+1}=U\psi^{n}
\end{equation}
\end{block}

%----------------------------------------------------------------------------------------

%----------------------------------------------------------------------------------------
%----------------------------------------------------------------------------------------

\end{column} % End of the first column

\begin{column}{\sepwid}\end{column} % Empty spacer column

\begin{column}{\twocolwid} % Begin a column which is two columns wide (column 2)

\begin{columns}[t,totalwidth=\twocolwid] % Split up the two columns wide column

\begin{column}{\onecolwid}\vspace{-.6in} % The first column within column 2 (column 2.1)

%----------------------------------------------------------------------------------------
%	ALGORITHM
%----------------------------------------------------------------------------------------
\begin{block}{Problem descriptions}
The problem to solve is the 2D TDSE (equation \eqref{SchrodPosBasis}) for the double slit experiment with the following restrictions:
	\begin{gather}
	V=0,\quad\Omega=\left[0,1\right]^{2},\quad\psi = 0\text{ on } \partial\Omega
	\end{gather}
	for a variety of initial conditions.
\end{block}
\begin{block}{Test case}
The method was tested with the 2D infinite square well (homogeneous Dirichlet boundary conditions), which has an analytical solution. The initial wavefunction is a superposition of the first two eigenstates.

\begin{figure}[H]
\includegraphics[width=0.8\linewidth]{placeholder.jpg}
\caption{Figure caption}
\end{figure}
\begin{table}[H]
	\begin{tabular}{c c c}
		\toprule
		\textbf{Methods} & \textbf{Measurement 1} & \textbf{Measurement 2}\\
		\midrule
		Method 1 & 0.0003262 & 0.562 \\
		Method 2 & 0.0015681 & 0.910 \\
		Method 3 & 0.0009271 & 0.296 \\
		\bottomrule
	\end{tabular}
	\caption{Table caption}
\end{table}

\end{block}

%----------------------------------------------------------------------------------------

\end{column} % End of column 2.1

\begin{column}{\onecolwid}\vspace{-.6in} % The second column within column 2 (column 2.2)

%----------------------------------------------------------------------------------------
%	METHODS
%----------------------------------------------------------------------------------------

\begin{block}{Double slit experiment}
The domain is similar to $\Omega=[-1,1]$ with the presence of a division at $x=0$ and two slits at around $y=0$ connecting the two halves. Homogeneous Dirichlet boundary conditions were used, as well as a plane wave initial condition $\exp\left(ikx\right)$.
\begin{figure}[H]
\includegraphics[width=\linewidth]{DoubleSlit0.png}
\caption{Initial conditions for double slit.}
\end{figure}
The wave travels to the right and hits the slits. A part is reflected, and the other goes through the slits and experience interference:
\begin{figure}[H]
	\includegraphics[width=\linewidth]{DoubleSlit115.png}
	\caption{Double slit solution.}
\end{figure}
\end{block}

%----------------------------------------------------------------------------------------

\end{column} % End of column 2.2

\end{columns} % End of the split of column 2 - any content after this will now take up 2 columns width

%----------------------------------------------------------------------------------------
%	IMPORTANT RESULT
%----------------------------------------------------------------------------------------

% \begin{alertblock}{Important Result}

% Lorem ipsum dolor \textbf{sit amet}, consectetur adipiscing elit. Sed commodo molestie porta. Sed ultrices scelerisque sapien ac commodo. Donec ut volutpat elit.

% \end{alertblock} 

%----------------------------------------------------------------------------------------

\begin{columns}[t,totalwidth=\twocolwid] % Split up the two columns wide column again

% \begin{column}{\onecolwid} % The first column within column 2 (column 2.1)

% %----------------------------------------------------------------------------------------
% %	MATHEMATICAL SECTION
% %----------------------------------------------------------------------------------------

% % \begin{block}{Multilevel RSB Algorithm}

% % \begin{itemize}
% % 	\item 4 steps.
% % \end{itemize}

% % % \begin{equation}
% % % E = mc^{2}
% % % \label{eqn:Einstein}
% % % \end{equation}

% % \end{block}

% %----------------------------------------------------------------------------------------

% \end{column} % End of column 2.1

% \begin{column}{\onecolwid} % The second column within column 2 (column 2.2)

%----------------------------------------------------------------------------------------
%	RESULTS
%----------------------------------------------------------------------------------------

% \begin{block}{Current status}

% \begin{center}
% \begin{tabular}{cc}
% \includegraphics[width=0.4\linewidth]{mesh_1.png} & \hfill & \includegraphics[width=0.4\linewidth]{mesh_2.png}
% \end{tabular}
% \end{center}

%  \end{block}

%----------------------------------------------------------------------------------------



% \end{column} % End of column 2.2

\end{columns} % End of the split of column 2

\end{column} % End of the second column

\begin{column}{\sepwid}\end{column} % Empty spacer column

\begin{column}{\onecolwid} % The third column

%----------------------------------------------------------------------------------------
%	CONCLUSION
%----------------------------------------------------------------------------------------
\begin{block}{}
	From a classical perspective, one would expect to detect particles in regions with the same shape as the slits, but this does not happen. This shows the inability of the paradigm of matter as only particles or only waves to explain the behavior of quantum-scale objects, and leads to the concept of wave-particle duality.
	
\end{block}
\begin{block}{Conclusion}
A numerical realization of the double slit experiment was presented, showing that the Time-Dependent Schr\"{o}dinger equation predicts wave-like behavior for objects in the quantum scale.\par
It was also shown that Crank-Nicolson is a reasonable timestepping scheme for this problem based on the error statistics and the conservation of probability.

\end{block}

%----------------------------------------------------------------------------------------
%	ADDITIONAL INFORMATION
%----------------------------------------------------------------------------------------

\begin{block}{Future Work}
In order to address many systems of physical importance, the current work could be modified to make a general solver for the TDSE. This process involves the following:
\begin{itemize}
\item Generalize boundary conditions.
\item Improve spatial resolution.
\item Use time-dependent potentials.
\item Implement in 3D.
\end{itemize}

\end{block}

%----------------------------------------------------------------------------------------
%	REFERENCES
%----------------------------------------------------------------------------------------

\begin{block}{References}

\nocite{*} % Insert publications even if they are not cited in the poster
\small{\bibliographystyle{unsrt}
\bibliography{bib.bib}\vspace{0.75in}}

\end{block}

%----------------------------------------------------------------------------------------
%	ACKNOWLEDGEMENTS
%----------------------------------------------------------------------------------------

% \setbeamercolor{block title}{fg=red,bg=white} % Change the block title color

% \begin{block}{Acknowledgements}

% \small{\rmfamily{Nam mollis tristique neque eu luctus. Suspendisse rutrum congue nisi sed convallis. Aenean id neque dolor. Pellentesque habitant morbi tristique senectus et netus et malesuada fames ac turpis egestas.}} \\

% \end{block}

%----------------------------------------------------------------------------------------
%	CONTACT INFORMATION
%----------------------------------------------------------------------------------------

% \setbeamercolor{block alerted title}{fg=black,bg=norange} % Change the alert block title colors
% \setbeamercolor{block alerted body}{fg=black,bg=white} % Change the alert block body colors

% \begin{alertblock}{Contact Information}

% \begin{itemize}
% \item Web: \href{http://www.university.edu/smithlab}{http://www.university.edu/smithlab}
% \item Email: \href{mailto:john@smith.com}{john@smith.com}
% \item Phone: +1 (000) 111 1111
% \end{itemize}

% \end{alertblock}

% \begin{center}
% \begin{tabular}{ccc}
% \includegraphics[width=0.4\linewidth]{logo.png} & \hfill & \includegraphics[width=0.4\linewidth]{logo.png}
% \end{tabular}
% \end{center}

%----------------------------------------------------------------------------------------

\end{column} % End of the third column

\end{columns} % End of all the columns in the poster

\end{frame} % End of the enclosing frame

\end{document}
